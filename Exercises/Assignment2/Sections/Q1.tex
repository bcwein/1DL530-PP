\section{Numerical Integration}

To numerically calculate a integral of a function $f$. We can
use the trapezoidal rule \cite{trapezoidal}. The trapezoidal rule 
approximates the area under a graph of a function as a trapezoid and 
calulates the area. This can be done for more and more trapezoids to increase
accuraccy but also coming at a computational cost. The trapezoidal rule can 
approximate a definate integral with riemann sums by splitting the interval 
$[a, b]$ such as $a < x_0 < x_1 < \dots < x_{N-1} < x_N$ and perform the 
following calculation

\begin{equation}
    \int_a^b f(x) \, dx \approx \frac{\Delta x}{2} \left(f(x_0) + 2f(x_1) + \cdots + 2f(x_{N-1}) + f(x_N)\right)
\end{equation}

The sum in this calculation does not depend on eachother and can therefore be
parallelised and calculated independently and summed up when they are complete.
This has been implemented in our code by creating a C++ program that takes in
the command line arguments $N$ for the no of threads and $T$ for the no of
trapezoids. All threads gets $\frac{N}{T}$ trapezoids besides the last unlucky
thread that gets $\frac{N}{T} + T \Mod N$ due to ease of implementation. All
the threads operate this function and work on a shared variable to calculate 
the sum, hence the mutexes.

\begin{lstlisting}[language=C++, caption=non-determinism.cpp]
void *calculateFactorial(void *conf)
{
  Config *cfg = (Config *)conf;
  double localResults;

  for (int i = 0; i < cfg->numTrapPerThread; i++)
  {
    int pos = cfg->startI * cfg->numTrapPerThread + i;
    if (pos == 0 || pos == numTrapezes)
    {
      continue;
    }
    localResults = localResults + 2 * function(a + ((pos)*w));
  }

  results_mutex.lock();
  results += localResults;
  results_mutex.unlock();

  pthread_exit(0);
}
\end{lstlisting}

