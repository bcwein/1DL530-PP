\section{Instructions}

For compiling the code that was provided in the assignment as well as the 
code that was edited for solving the exercises, the makefile provided was used.
In some cases, the programs would not compile with -std=c++11 flag. This was mentioned
in the lectures. If that is the case. Replace the flag with -std=c++0x flag.

We have used \textit{siegnbahn.it.uu.se} for checking that the code runs.
The code is structured in the following way:

\begin{itemize}
    \item Q1
        \begin{itemize}
            \item \textit{question1.cpp}
            \item Makefile
        \end{itemize}
    \item Q2
        \begin{itemize}
            \item \textit{question2.cpp}
            \item Makefile
        \end{itemize}
    \item Q4
        \begin{itemize}
            \item Coarse-grained locking.
                \begin{itemize}
                    \item \textit{benchmark\_example.cpp}
                    \item \textit{benchmark.hpp}
                    \item \textit{sorted\_list.hpp}
                    \item Makefile
                \end{itemize}
            \item Fine-grained locking.
                \begin{itemize}
                    \item Same as above
                \end{itemize}
            \item Coarse-grained locking TATAS
                \begin{itemize}
                    \item Same as above
                \end{itemize}
            \item fine-grained locking TATAS
                \begin{itemize}
                    \item Same as above
                \end{itemize}
            \item fine-grained locking using scalable queue lock
                \begin{itemize}
                    \item Same as above
                \end{itemize}
        \end{itemize}
\end{itemize}

There is a makefile for every question and subquestion. To check a question
just navigate to the corresponding code and run:

\begin{center}
    \textit{Make PROG=\textbf{filename}}
\end{center}

Where \textit{\textbf{filename}} is the name of the C++ file to compile.