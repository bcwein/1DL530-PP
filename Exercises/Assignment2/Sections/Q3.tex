\section{Mutual Exclusion Algorithms}

\subsection{Question 1}

\begin{center}
    \textit{Does this protocol satisfy mutual exclusion?}
\end{center}

Yes (in a dangerously fragile manner since it is not deadlock free. See question 
3). An example of this can be see with two threads.

\begin{enumerate}
    \item Thread 1 runs the 'lock' method. This in turn sets 'turn' to its ID 
    and sets 'busy' to true. The lock method then returns and Thread 1 goes into 
    the critical section.
    \item Thread 2 now attempts to go into the critical section. Thread 2 runs 
    the 'lock' method. 'turn' is now set to Thread2's ID, and it loops as it 
    fails the 'busy=true' test.
    \item Thread 1 now completes the critical section, and thus runs the 'unlock'
    method. This sets 'busy' to false.
    \item Thread 2 now fails the 'busy=true' test and thus exits the loop. 
    It sets 'busy=true' and fails the 'while (turn != me)' test, thus exiting 
    the lock method.
    \item Thread 2 now enters the critical section.
\end{enumerate}

\subsection{Question 2}

\begin{center}
    \textit{Is this protocol starvation-free?}
\end{center}

Yes in the sense that no priority is given to any of the waiting threads. 
And thus the quickest thread to set 'turn=me' when 'busy = false' will get the 
lock (could this leave one thread waiting for longer than others? Yes, but that 
would be due to the thread being slow in the race.) An example of this can be 
seen when looking at three threads (and ignores deadlocks).

\begin{enumerate}
\item Thread 1 arrives to the critical section first, it runs the method 'lock', 
sets 'busy' to true and returns from the method.
\item Thread 2 arrives at the critical section and attempts to run the method 
'lock'. It sets 'turn=me' and passes the 'busy=false' check thus looping and 
waiting for the lock
\item Thread 3 arrives at the critical section and it too attempts to run the 
method 'lock'. It sets 'turn=me' and passes the 'busy=false' check thus looping 
and waiting for the lock.
\item Therefore Thread 2 and 3 are both waiting on the lock to release 
(i.e 'busy=false') and are equally setting 'turn' to their IDs.
\item Thread 1 completes the critical section and sets 'busy=false'
\item Thread 3 wins the race and fails the 'busy=false' check before thread 2 
can change the 'turn' variable. Thus Thread 3 exits the loop and holds the lock.
\item Thread 1 now attempts to enter the critical section once again and it too 
attempts to run the method 'lock'. It sets 'turn=me' and passes the 'busy=false' 
check thus looping and waiting for the lock.
\item At this point we now have Thread 1 and Thread 2 waiting on the lock to 
release and are equally setting 'turn' to their IDs.
\item Thread 3 completes the critical section and releases the lock
\item The quicker thread between 1 and 2 grabs the lock and the other lock waits.
\item This continues as more threads attempt to enter the critical section.
\end{enumerate}

\subsection{Question 3}

\begin{center}
    \textit{Is this protocol deadlock-free?}
\end{center}

No. Due to the do while approached used, the 'turn' variable can be changed 
between setting 'busy=true' and the check 'turn != me'. This will occur when two 
threads try to execute the 'lock' method simultaneously

\begin{enumerate}
\item Thread 1 and Thread 2 attempt to enter the critical section.
\item Thread 1 is slightly quicker and sets 'turn=me' and 'busy=true'
\item At this point Thread 2 enters the second do-while loop and sets 'turn = me' .
\item Thread 2 will now constantly pass the 'while (busy)' check (leaving it to loop).
\item While Thread 1 will constantly pass the 'while ( turn !=me )' check (leaving it to loop).
\item Thus both threads will never return from the 'lock' as each depends on the other, thus leading to a deadlock
\end{enumerate}