\section{Dining Philosophers}

Dining philosophers is a known example problem in computer science formulated
by Edsger Dijkstra in 1965 to illustrate a problem that can arise from concurrent
programs. In the assignment there has been provided a program which simulates
the dining philosophers problem. After compilation and a few trial runs we get
a output similar to this which halts:

\begin{lstlisting}[caption=Output from the dining philosophers program.]
Philosopher 0 is thinking.
Philosopher 0 picked up her left fork.
Philosopher 0 picked up her right fork.
Philosopher 0 is eating.
Philosopher 0 is putting down her right fork.
Philosopher 0 is putting down her left fork.
Philosopher 0 is thinking.
Philosopher 0 picked up her left fork.
Philosopher 0 picked up her right fork.
Philosopher 0 is eating.
Philosopher 0 is putting down her right fork.
Philosopher 0 is putting down her left fork.
Philosopher 0 is thinking.
Philosopher 0 picked up her left fork.
Philosopher 0 picked up her right fork.
Philosopher 0 is eating.
Philosopher 1 is thinking.
Philosopher 0 is putting down her right fork.
Philosopher 0 is putting down her left fork.
Philosopher 0 is thinking.
Philosopher 0 picked up her left fork.
Philosopher 1 picked up her left fork.
\end{lstlisting}

The program above had two philosophers and two forks. What has happened is that 
philosopher 0 picks up her left fork and philosopher 1 picks up her left fork.
This means that both philosophers are at step 2 and will wait indefinitely for
the other to put down their fork. This means that we have a deadlock, a state in 
which each member waits for the other member, including itsef, to take action.

To solve the dining philosophers problem, we have implemented the Arbitrator 
solution. This means that we introduced a mutex that can be thought of as a 
waiter that does not allow a philosopher to pick up both forks or none. This 
comes at the cost of lost parallellism as well as adding a new mutex.

The change that we did is shown in the listing below:

\begin{lstlisting}[language=C++, caption=Arbitrator/Waiter code modification]
    arbitrator.lock();
    left->lock();
    std::cout << "Philosopher " << n << " picked up her left fork." << std::endl;

    right->lock();
    std::cout << "Philosopher " << n << " picked up her right fork." << std::endl;
    arbitrator.unlock();
\end{lstlisting}