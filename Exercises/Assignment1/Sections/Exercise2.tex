\section{Shared-Memory Concurrency}

We have, as stated in the assignment description, three threads that 
operate on a shared integer variable named \textbf{x}

The first thread \textbf{t1} operates the following code

\begin{lstlisting}[language=C++, caption=Thread 1 source code]
void inc()
{
  bool r = true;
  while (r)
    {
      mutex.lock();
      ++x;
      r = run;
      mutex.unlock();
    }
}
\end{lstlisting}

This thread runs an 'infinite' loop on the boolean variable \textit{r} which is
true until the boolean value r is equal to \textit{run}, which we will see later
in the main function. Thread 1 raises it mutual exclusion flag before it increments
\textit{x} and assigns \textit{run} to \textit{r} before it lowers its flag.

The second thread \textbf{t2} operates the following code

\begin{lstlisting}[language=C++, caption=Thread 2 source code]
void dec()
{
  bool r = true;
  while (r)
    {
      mutex.lock();
      --x;
      r = run;
      mutex.unlock();
    }
}
\end{lstlisting}
This thread does the same thing as \textit{t1} but opposite. It loops infinitely
and raises it flag when decrementing \textit{x} and assigning \textit{run} to 
\textit{r}.  

The last thread \textbf{t3} operates the following code

\begin{lstlisting}[language=C++, caption=Thread 3 source code]
void print()
{
  bool r = true;
  while (r)
    {
      mutex.lock();
      std::cout << x << std::endl;
      r = run;
      mutex.unlock();
    }
}
\end{lstlisting}

This thread also loops infinitely on \textit{r} until that gets assigned false
by \textit{run}. It raises its mutual exclusion flag before outputting the value
of \textit{x} and when assigning \textit{run} to \textit{r}.

Lastly, lets see the main method, how it uses these threads and what we would
expect as output. 

\begin{lstlisting}[language=C++, caption=Main method]
int main()
{
  std::thread t1(inc);
  std::thread t2(dec);
  std::thread t3(print);

  std::this_thread::sleep_for(std::chrono::seconds(1));

  mutex.lock();
  run = false;
  mutex.unlock();

  t1.join();
  t2.join();
  t3.join();

  return 0;
}
\end{lstlisting}

We see that the threads are started and the program sleeps for 1 second.
In that time, all three threads are operating on the same shared variable.
Given how the mutual exclusion flags are handled, when one thread loops, another
thread may be given access to the variable. Since the order of what threads will
get access and for what time during the second of sleep is non-deterministic, the
variable \textit{x} should also have a non-deterministic value when outputted.

After five trial runs, the last outputted value of x was: $57031$, $-44762$, 
$-45637$, $220150$ and $11428$. So we see that the final value of x can be 
considered stochastic and non-deterministic. 