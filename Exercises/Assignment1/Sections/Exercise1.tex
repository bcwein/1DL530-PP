\section{Concurrency and Non-Determinism}

The code provided on studium is shown below

\begin{lstlisting}[language=C++, caption=non-determinism.cpp]
#include <iostream>
#include <mutex>
#include <thread>


std::mutex mutex;

void loop(int n)
{
  mutex.lock();
  std::cout << "Task " << n << " is running." << std::endl;
  mutex.unlock();

  mutex.lock();
  std::cout << "Task " << n << " is terminating." << std::endl;
  mutex.unlock();
}

int main()
{
  std::thread t1(loop, 1);
  std::thread t2(loop, 2);
  std::thread t3(loop, 3);
  std::thread t4(loop, 4);
  std::thread t5(loop, 5);
  std::thread t6(loop, 6);
  std::thread t7(loop, 7);
  std::thread t8(loop, 8);

  t1.join();
  t2.join();
  t3.join();
  t4.join();
  t5.join();
  t6.join();
  t7.join();
  t8.join();

  return 0;
}
\end{lstlisting}

And when we compile this program as told in the exercise, we get an output
similar to this:

\begin{lstlisting}
Task 1 is running.
Task 3 is running.
Task 3 is terminating.
Task 1 is terminating.
Task 4 is running.
Task 5 is running.
Task 6 is running.
Task 7 is running.
Task 7 is terminating.
Task 5 is terminating.
Task 4 is terminating.
Task 6 is terminating.
Task 2 is running.
Task 2 is terminating.
Task 8 is running.
Task 8 is terminating.
\end{lstlisting}

But exactly what tasks starts runnning and terminating in what order is
observed to be stochastic. This is known as \textbf{non-determinism} that is 
known to arise when we have concurrent programs. In the observed output which 
task that will start to run and in what order is random. The non-determinism
of concurrent programs is important to take into account as this could lead to
the final output also becoming non-deterministic. Therefore we need mutual 
exclusion flags to ensure that the concurrent program outputs the correct solution
for the problem that it is solving. In the program above this mutual exlustion
flag is raised when outputting that a task is running and when it is terminating.
Mutual exlusion flags solves some problems but will also lead to other problems
which we will see later.

