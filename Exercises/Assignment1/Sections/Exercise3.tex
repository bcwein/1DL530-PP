\section{Race Conditions and Data Races}

A race condition occurs when a program depends on the timing of one or more
processes to function correctly. A data race is a special case of race condition
where a shared variable is being operated on by several threads. The first 
program \textit{non-determinism.cpp} does have a race condition which gives 
different outputs based on the order of the threads. No data race exists though 
as there is no shared variable that the threads operate on. 

The second program, \textit{shared-variable.cpp} does produce different outputs
dependent on the sequence of execution on the threads and we therefore
have a race condition. The operations on the shared variable are atomic and 
therefore the program does not have a data race on the shared variable.
This has been verified by compiling the program with 'fsanitize=thread' and 
running it.

In summary: Both programs are examples of race conditions, none of them are 
examples of data races due to atomic operations on shared data.