\section{Race Conditions and Data Races}

A race condition occurs when a program depends on the timing of one or more
processes to function correctly. A data race is a special case of race condition
where a shared variable is being operated on by several threads. The first 
program \textit{non-determinism.cpp} could potentially produce a race condition and
the second program \textit{shared-variable.cpp} could produce a data race.

This depends on the desired operation of the above mentioned programs. The
first thread reports which task is running and terminating in what order. Since 
the print statements are in mutual exclusive sections a error in the print 
statement will not happen. Because of that i do not consider 
\textit{non-determinism.cpp} to produce a race condition. If these mutual exlusive 
sections were not implemented, another thread could take over before in the time 
that the print is finished and a race condition is produced.

The second program, \textit{shared-variable.cpp} does produce different outputs
dependent on the sequence of execution on the threads but i consider that to be 
a part of the operation of the program to demonstrate how the access of the threads
is random for each time the program is ran. The operations on 
the shared variable are atomic and will not produce errors and invalid output
that does not reflect the behaviour of the program. I therefore do not consider
\textit{shared-variable.cpp} to produce a data race.